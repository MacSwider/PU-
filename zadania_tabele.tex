\documentclass[12pt, letterpaper, titlepage]{article}
\usepackage{xcolor}
\usepackage[left=3.5cm, right=2.5cm, top=2.5cm, bottom=2.5cm]{geometry}
\usepackage[MeX]{polski}
\usepackage[utf8]{inputenc}
\usepackage{graphicx}
\usepackage{enumerate}
\usepackage{amsmath} %pakiet matematyczny
\usepackage{amssymb} %pakiet dodatkowych symboli
\usepackage{caption}
\title{Zadania z Tabelami}
\author{Maciej Świder (nr.indeksu 169370 lub 266026)}
\date{Październik 2022}
\begin{document}
\maketitle


\begin{table}[h] %zadanie_1%
\centering\caption{Tabela do zadania 1}
\begin{tabular}{l|lll}
\hline
\hline
Pacjent & Ból brzucha & Temperatura ciała & Operacja\\
\hline
u1 & Mocny & Wysoka & Tak \\
\hline
u2 & Średni & Wysoka & Tak \\
\hline
u3 & Mocny & Średnia & Tak \\
\hline
u4 & Mocny & Wysoka & Tak \\
\hline
u5 & Średni & Średnia & Tak \\
\hline
u6 & Średni & Średnia & Nie \\
\hline
u7 & Mały & Wysoka & Nie \\
\hline
u8 & Mały & Niska & Nie \\
\hline
u9 & Mocny & Niska & Nie \\
\hline
u10 & Mały & Średnia & Nie \\
\hline
\hline

\end{tabular}
\end{table}

\begin{table}[h] %zadanie_2%
\centering\caption{Tabela do zadania 2}
\begin{tabular}{l|l|l|l}
\hline
\hline
Typ bramki & Wejście-A & Wejście-B & Wyjście \\
\hline 
NOT & 0 & - & 1\\
\hline
 & 1 & - & 0\\
\hline
AND & 0 & 0 & 0\\
\hline
 & 0 & 1 & 0\\
\hline
 & 1 & 0 & 0\\
\hline
 & 1 & 1 & 1\\
\hline
NAND & 0 & 0 &1\\
\hline
 & 0 & 1 &1\\
\hline
 & 1 & 0 &1\\
\hline
 & 1 & 1 &0\\
\hline
OR & 0 & 0 &0\\
\hline
 & 0 & 1 &1\\
\hline
 & 1 & 0 &1\\
\hline
 & 1 & 1 &1\\
\hline
NOR & 0 & 0 &1\\
\hline
 & 0 & 1 &0\\
\hline
 & 1 & 0 &0\\
\hline
 & 1 & 1 &0\\
\hline
XOR & 0 & 0 &0\\
\hline
 & 0 & 1 &1\\
\hline
 & 1 & 0 &1\\
\hline
 & 1 & 1 &0\\
\hline
\hline


\end{tabular}
\end{table}

\end{document}