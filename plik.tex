\documentclass[12pt, letterpaper, titlepage]{article}
\usepackage[left=3.5cm, right=2.5cm, top=2.5cm, bottom=2.5cm]{geometry}
\usepackage[MeX]{polski}
\usepackage[utf8]{inputenc}
\usepackage{graphicx}
\usepackage{enumerate}
\usepackage{amsmath} %pakiet matematyczny
\usepackage{amssymb} %pakiet dodatkowych symboli
\title{Pierwszy dokument LaTeX czyli numeracje,teksty i przepisy}
\author{Maciej Świder (nr.indeksu 169370)}
\date{Październik 2022}
\begin{document}
\maketitle
\section{Wstęp}
Jeden z wielu studentów postawionych przez konkretnym zadaniem do wykonania.
\begin{quote}
"Jak coś jest głupie i działa to niej jest głupie"
\end{quote}
\section{Akcja-Numeracja}
Jednym z celów zadania było utworzenie zagnieżdżeń i numeracji
\begin{enumerate}[I]
\item Punkt Pierwszy
\item Punkt Drugi
\item Punkt Trzeci
\end{enumerate}

\section{American Pancakes}
Robiłem dzisiaj na śniadanie. Szczerze polecam
\subparagraph{Składniki} :		 \newline
-30g Masła \newline
-40g Cukru \newline
-2 jajka \newline
-5 g proszku do pieczenia \newline
-400ml mleka \newline
-400g mąki \newline
\subparagraph{Krok po Kroku}
\begin{enumerate}

\item Roztopione masło wymieszań najpierw z cukrem a potem z jajkami
\item Dolać mleko i dosypać proszek do pieczenia. Dokładnie wymieszać
\item Stopniowo dodawać mąkę, cały czas mieszając. Ciasto powinno być gęste
\item Piec na średnim ogniu do uzyskania złotego koloru
\end{enumerate}
\subparagraph{Protip od autora}
\begin{enumerate}[*]
\item \textit{Dla uzyskania bardziej puszystego ciasta można białko wymieszać mikserem a potem ostrożnie wymieszać je z resztą ciasta}
\item \textit{Dla jednolitego koloru najlepiej używać patelni teflonowej niczym nie posmarowanej}
\end{enumerate}



\end{document}